%%%%%%%%%%%%%%%%%%%%%%%%%%%%%%%%%%%%%%%%%%%%%%%%%%%%%%%%%%%%%%%%%%%%%%%%%%%%%%%
\section{Spin symmetry operation search}

A spin arrangement is represented by the following four objects:
(1) basis vectors of its lattice $\bm{A} = (\bm{a}_{1}, \bm{a}_{2}, \bm{a}_{3})$,
(2) an array of point coordinates of sites in its unit cell $\bm{X} = (\bm{x}_{1}, \cdots, \bm{x}_{N})$,
(3) an array of atomic types of sites in its unit cell $\bm{T} = (t_{1}, \cdots, t_{N})$,
and (4) an array of magnetic moments of sites in its unit cell $\bm{M} = (\bm{m}_{1}, \cdots, \bm{m}_{N})$,
where $N$ is the number of sites in the unit cell.

First, we consider a crystal structure $(\bm{A}, \bm{X}, \bm{T})$ obtained by ignoring magnetic moments of $(\bm{A}, \bm{X}, \bm{T}, \bm{M})$.
A space group of $(\bm{A}, \bm{X}, \bm{T})$ is written as a stabilizer subgroup of $\mathrm{E}(3)$ that preserves $(\bm{A}, \bm{X}, \bm{T})$:
\begin{align}
    \mathcal{S}
        &\coloneqq \set{
                g = ( \overline{\bm{R}}, \overline{\bm{v}} )_{\bm{A}} \in \mathrm{E}(3)
            }{
                \begin{array}{l}
                    \overline{\bm{R}} \in \mathbb{Z}^{3 \times 3} \\
                    \exists \sigma_{g} \in \mathfrak{S}_{N}, \forall i, \\
                    \overline{\bm{R}} \bm{x}_{i} + \overline{\bm{v}} \equiv \bm{x}_{\sigma_{g}(i)} \, (\mathrm{mod} \, 1) \\
                    t_{i} = t_{\sigma_{g}(i)}
                \end{array}
            },
\end{align}
where $\mathfrak{S}_{N}$ is a symmetric group of degree $N$ and $( \overline{\bm{R}}, \overline{\bm{v}} )_{\bm{A}} \coloneqq ( \bm{A}\overline{\bm{R}}\bm{A}^{-1}, \bm{A}\overline{\bm{v}} )$ \footnote {
  The condition $\overline{\bm{R}} \in \mathbb{Z}^{3 \times 3}$ corresponds to Spglib's convention to omit symmetry operations that do not preserve a supercell.
}.

Let $\bm{A}_{\mathcal{S}}$ be one of primitive basis vectors for $(\bm{A}, \bm{X}, \bm{T})$.
We write a translation subgroup formed by basis vector $\bm{A}$ as
\begin{align}
  \mathcal{T}_{\bm{A}} \coloneqq \set{ (\bm{E}, \bm{n})_{\bm{A}} }{ \bm{n} \in \mathbb{Z}^{3} }.
\end{align}
Then we consider a finite coset decomposition of $\mathcal{S}$,
\begin{align}
  \mathcal{T}_{\bm{A}_{\mathcal{S}}}
    &= \bigsqcup_{ \overline{ \bm{t} }^{\mathcal{S}} }
      \left( \bm{E}, \overline{ \bm{t} }^{\mathcal{S}} \right)_{ \bm{A}_{\mathcal{S}} }
      \mathcal{T}_{\bm{A}} \\
  \mathcal{S}
    &= \bigsqcup_{
          \overline{\bm{R}}^{\mathcal{S}} \in \bm{A}_{\mathcal{S}}^{-1} \mathcal{P}(\mathcal{S}) \bm{A}_{\mathcal{S}}
      }
      \bigsqcup_{
          \left( \bm{E}, \overline{ \bm{t} }^{\mathcal{S}} \right)_{ \bm{A}_{\mathcal{S}} } \mathcal{T}_{\bm{A}}
          \in \mathcal{T}_{\bm{A}_{\mathcal{S}}} / \mathcal{T}_{\bm{A}}
      }
      \left(
        \overline{\bm{R}}^{\mathcal{S}},
        \overline{ \bm{v} }^{\mathcal{S}}(\overline{\bm{R}}^{\mathcal{S}})
      \right)_{ \bm{A}_{\mathcal{S}} }
      \left( \bm{E}, \overline{ \bm{t} }^{\mathcal{S}} \right)_{ \bm{A}_{\mathcal{S}} }
      \mathcal{T}_{\bm{A}},
\end{align}
where $\overline{ \bm{v} }^{\mathcal{S}}(\overline{\bm{R}}^{\mathcal{S}})$ is a translation part for a rotation $\bm{A}_{\mathcal{S}} \overline{\bm{R}}^{\mathcal{S}} \bm{A}_{\mathcal{S}}^{-1}$.
Note that a point group of $\mathcal{S}$ is $\mathcal{P}(\mathcal{S}) = \{ \bm{A}_{\mathcal{S}} \overline{\bm{R}}^{\mathcal{S}} \bm{A}_{\mathcal{S}}^{-1} \}_{ \overline{\bm{R}}^{\mathcal{S}} }$.

A spin space group of $(\bm{A}, \bm{X}, \bm{T}, \bm{M})$ is written as a stabilizer subgroup of $\mathrm{E}(3) \times \mathrm{O}(3)$ that preserves $(\bm{A}, \bm{X}, \bm{T}, \bm{M})$:
\begin{align}
  \mathcal{G}
    &\coloneqq \set{
      (g, \bm{W}) \in \mathrm{E}(3) \times \mathrm{O}(3)
    }{
        \begin{array}{l}
          \exists \sigma_{g} \in \mathfrak{S}_{N}, \forall i, \\
          g \bm{x}_{i} \equiv \bm{x}_{\sigma_{g}(i)} \, (\mathrm{mod} \, 1) \\
          t_{i} = t_{\sigma_{g}(i)} \\
          \bm{W} \bm{m}_{i} = \bm{m}_{\sigma_{g}(i)}
        \end{array}
    } \\
    &= \set{
      (g, \bm{W}) \in \mathcal{S} \times \mathrm{O}(3)
    }{
        \begin{array}{l}
          \exists \sigma_{g} \in \mathfrak{S}_{N}, \forall i, \\
          g \bm{x}_{i} \equiv \bm{x}_{\sigma_{g}(i)} \, (\mathrm{mod} \, 1) \\
          t_{i} = t_{\sigma_{g}(i)} \\
          \bm{W} \bm{m}_{i} = \bm{m}_{\sigma_{g}(i)}
        \end{array}
    } \\
  \mathcal{D} &\coloneqq \mathcal{D}(\mathcal{G}) \\
  \mathcal{F} &\coloneqq \mathcal{F}(\mathcal{G}).
\end{align}

\subsection{Spin-only group search}

\begin{align}
  \mathcal{P}_{\mathrm{so}}
    &\coloneqq \mathcal{P}_{\mathrm{so}}(\mathcal{G})
    = \set{
        \bm{W} \in \mathrm{O}(3)
      }{
        \bm{W} \bm{m}_{i} = \bm{m}_{i} \,(\forall i)
      }
\end{align}

Consider a moment of inertia tensor of $\bm{M}$,
\begin{align}
  \bm{N} \coloneqq \sum_{i=1}^{N} \bm{m}_{i} \otimes \bm{m}_{i}.
\end{align}
Because $\bm{N}$ is a symmetric semi-definite matrix, we can consider its eigen decomposition,
\begin{align}
  \bm{N} = \sum_{r=1}^{3} \sigma_{r} \hat{\bm{n}}_{r} \otimes \hat{\bm{n}}_{r},
\end{align}
where $\sigma_{1} \geq \sigma_{2} \geq \sigma_{3} \geq 0$ and $\{ \hat{\bm{n}}_{r} \}_{r=1}^{3}$ are orthonormal.

\subsubsection{Nonmagnetic spin arrangement}

When all magnetic moments are zero, the spin arrangement is called \term{nonmagnetic}.
A spin-only group of a nonmagnetic spin arrangement is $\mathcal{P}_{\mathrm{so}} = \mathrm{O}(3)$.
In practice, we compare magnetic moments with a tolerance $\epsilon$,
\begin{align}
  \bm{m}_{i} = \bm{0} \rightarrow \norm{ \bm{m}_{i} }_{2} < \epsilon.
\end{align}

\subsubsection{Collinear spin arrangement}

When a spin arrangement is not nonmagnetic and all magnetic moments are parallel or antiparallel, the spin arrangement is called \term{collinear}.
Let $\hat{\bm{n}}$ be a direction parallel or antiparallel to the magnetic moments.
Let $\hat{\bm{n}}'$ be one of the directions perpendicular to the magnetic moments.
A spin-only group of a collinear spin arrangement is
\begin{align*}
  \mathcal{P}_{\mathrm{so}}
    = \set{ R_{\theta \hat{\bm{n}}} m_{\hat{\bm{n}}'}^{l} }{ 0 \leq \theta < 2 \pi, l=0, 1 }.
\end{align*}
Here $R_{\theta \hat{\bm{n}}}$ is a rotation along axis $\hat{\bm{n}}$ by $\theta$.
$m_{\hat{\bm{n}}'}$ is a mirror operation perpendicular to $\hat{\bm{n}}'$.

When a spin arrangement is collinear, the eigenvector $\hat{\bm{n}}_{1}$ should be parallel or antiparallel to all magnetic moments.
We can check it with tolerance as
\begin{align}
  \forall \theta \in [0, 2 \pi ), l \in \{ 0, 1 \}, R_{\theta \hat{\bm{n}}_{1}} m_{\hat{\bm{n}}_{1}'}^{l} \bm{m}_{i} = \bm{m}_{i}
  \rightarrow
  2 \norm{ \bm{m}_{i} - (\bm{m}_{i} \cdot \hat{\bm{n}}_{1})\hat{\bm{n}}_{1} }_{2} < \epsilon.
\end{align}

\subsubsection{Coplanar spin arrangement}

When a spin arrangement is not collinear and all magnetic moments are perpendicular to $\hat{\bm{n}}$, the spin arrangement is called \term{coplanar}.
A spin-only group of a coplanar spin arrangement is $\mathcal{P}_{\mathrm{so}} = \{ 1, m_{\hat{\bm{n}}} \}$.

When a spin arrangement is coplanar, the eigenvector $\hat{\bm{n}}_{3}$ should be perpendicular to all magnetic moments.
We can check it with tolerance as
\begin{align}
  m_{\hat{\bm{n}}_{3}} \bm{m}_{i} = \bm{m}_{i}
  \rightarrow
  2 \norm{ (\bm{m}_{i} \cdot \hat{\bm{n}}_{3}) \hat{\bm{n}}_{3}}_{2} < \epsilon.
\end{align}

\subsubsection{Noncoplanar spin arrangement}

When a spin arrangement is not coplanar, the spin arrangement is called \term{noncoplanar}.
A spin-only group of a noncoplanar spin arrangement is $\mathcal{P}_{\mathrm{so}} = \{ 1 \}$.

\subsection{Translation subgroup of maximal space subgroup}

\begin{figure}
  \centering
  \begin{tikzpicture}
    \coordinate (TA) at (0, 0);
    \coordinate (TAD) at (0, 2);
    \coordinate (TAS) at (0, 4);
    \coordinate (SD) at (2, 3);
    \coordinate (mid) at (2, 5);
    \coordinate (S) at (4, 6);

    \node at (TA) [left] {$\mathcal{T}_{\bm{A}}$};
    \node at (TAD) [left] {$\mathcal{T}_{\bm{A}_{\mathcal{D}}}$};
    \node at (TAS) [left] {$\mathcal{T}_{\bm{A}_{\mathcal{S}}}$};
    \node at (SD) [right] {$\mathcal{S}_{\mathcal{D}}$};
    \node at (S) [right] {$\mathcal{S}$};

    \fill[black] (TA) circle (2pt);
    \fill[black] (TAD) circle (2pt);
    \fill[black] (TAS) circle (2pt);
    \fill[black] (SD) circle (2pt);
    \fill[black] (mid) circle (2pt);
    \fill[black] (S) circle (2pt);

    \draw (TA) -- (TAD);
    \draw (TAD) -- (TAS);
    \draw (TAD) -- (SD);
    \draw (TAS) -- (mid);
    \draw (SD) -- (mid);
    \draw (mid) -- (S);
  \end{tikzpicture}
  \caption{\label{fig:translation_subgroup}Group-subgroup diagram of translation subgroups derived from spin space group.}
\end{figure}

We write the primitive basis vectors of $\mathcal{T}(\mathcal{D})$ as $\bm{A}_{\mathcal{D}}$.
First, we search for $\mathcal{T}(\mathcal{D}) = \mathcal{T}_{\bm{A}_{\mathcal{D}}} = \set{ (\bm{E}, \bm{v}) }{ (\bm{E}, (\bm{E}, \bm{v})) \in \mathcal{G} }$.
The group-subgroup relationships of translation subgroups are shown in Fig.~\ref{fig:translation_subgroup}.
The two basis vectors $\bm{A}$ and $\bm{A}_{\mathcal{S}}$ are related by an integer matrix as
\begin{align}
  \bm{A} = \bm{A}_{\mathcal{S}} \bm{U} \quad (\bm{U} \in \mathbb{Z}^{3 \times 3}).
\end{align}
Since $\mathcal{T}_{\bm{A}_{\mathcal{D}}}$ is a subgroup of $\mathcal{T}_{\bm{A}_{\mathcal{S}}}$, we only need to check finite coset representatives of $\mathcal{T}_{\bm{A}} / \mathcal{T}_{\bm{A}_{\mathcal{S}}}$ \footnote{
  The transformation matrix $\bm{U}$ can be calculated from centerings in $\mathcal{T}_{\bm{A}_{\mathcal{D}}} / \mathcal{T}_{\bm{A}_{\mathcal{S}}}$ and basis vectors of $\bm{A}_{\mathcal{S}}$.
  Primitive basis vectors of $\mathcal{T}_{\bm{A}_{\mathcal{D}}}$ can be taken from three nonzero vectors from the Hermite normal from of a matrix formed by these centerings and basis vectors of $\bm{A}_{\mathcal{S}}$.
}.
We can find translations in $\mathcal{T}_{\bm{A}_{\mathcal{D}}}$ as
\begin{align}
  \mathcal{T}_{\bm{A}_{\mathcal{S}}}
    &= \set{
      \left( \bm{E}, \overline{ \bm{t} }^{\mathcal{S}} \right)_{ \bm{A}_{\mathcal{S}} }
      }{
        \begin{array}{l}
          \exists \sigma \in \mathfrak{S}_{N}, \forall i, \\
          \bm{x}_{i} + \bm{U}^{-1} \overline{ \bm{t} }^{\mathcal{S}} \equiv \bm{x}_{\sigma(i)} \quad (\mathrm{mod}\, 1) \\
          t_{i} = t_{\sigma(i)} \\
          \bm{m}_{i} = \bm{m}_{\sigma(i)}
        \end{array}
      }
      \times \mathcal{T}_{\bm{A}_{\mathcal{D}}}.
\end{align}
Then, we take one of primitive basis vectors $\bm{A}_{\mathcal{D}}$ of with
\begin{align}
  \bm{A}_{\mathcal{D}}
    &= \bm{A}_{\mathcal{S}} \bm{V} \quad (\bm{V} \in \mathbb{Z}^{3 \times 3}).
\end{align}

\subsection{Spin translation group search}

\begin{figure}
  \centering
  \begin{tikzpicture}
    \coordinate (Pso) at (0, 0);
    \coordinate (PsoTAD) at (0, 2);
    \coordinate (PsoTAS) at (0, 4);
    \coordinate (Gst) at (2, 3);
    \coordinate (mid) at (2, 5);
    \coordinate (O3TAS) at (4, 6);

    \node at (Pso) [left] {$\mathcal{P}_{\mathrm{so}}$};
    \node at (PsoTAD) [left] {$\mathcal{P}_{\mathrm{so}} \times \mathcal{T}_{\bm{A}_{\mathcal{D}}}$};
    \node at (PsoTAS) [left] {$\mathcal{P}_{\mathrm{so}} \times \mathcal{T}_{\bm{A}_{\mathcal{S}}}$};
    \node at (Gst) [right] {$\mathcal{G}_{\mathrm{st}}$};
    \node at (O3TAS) [above] {$\mathrm{O}(3) \times \mathcal{T}_{\bm{A}_{\mathcal{S}}}$};

    \fill[black] (Pso) circle (2pt);
    \fill[black] (PsoTAD) circle (2pt);
    \fill[black] (PsoTAS) circle (2pt);
    \fill[black] (Gst) circle (2pt);
    \fill[black] (mid) circle (2pt);
    \fill[black] (O3TAS) circle (2pt);

    \draw (Pso) -- (PsoTAD);
    \draw (PsoTAD) -- (PsoTAS);
    \draw (PsoTAD) -- (Gst);
    \draw (Gst) -- (mid);
    \draw (PsoTAS) -- (mid);
    \draw (mid) -- (O3TAS);
  \end{tikzpicture}
  \caption{\label{fig:spin_translation_group}Group-subgroup diagram of spin translation group.}
\end{figure}

For coset representatives $(\bm{E}, \bm{v}) \mathcal{T}_{\bm{A}_{\mathcal{D}}} \in \mathcal{T}_{\bm{A}_{\mathcal{S}}} / \mathcal{T}_{\bm{A}_{\mathcal{D}}}$, we search for $\bm{W}_{\bm{v}}$ by solving a Procrustes problem as shown in Appendix~\ref{appx:procrustes}.
% Number of generators for $\{ \bm{W}_{\bm{v}} \}$ can be determined from the Smith normal form of $\bm{V}$.
The group-subgroup relationships for the spin translation group $\mathcal{G}_{\mathrm{st}}$ are shown in Fig.~\ref{fig:spin_translation_group}.

\begin{align}
  \mathcal{G}_{\mathrm{st}}
    &=
      \bigsqcup_{ \overline{\bm{v}}^{\mathcal{D}} }
      \bigsqcup_{
        (\bm{E}, \overline{\bm{t}}^{\mathcal{D}})_{\bm{A}_{\mathcal{D}}} \mathcal{T}_{\bm{A}}
          \in \mathcal{T}_{\bm{A}_{\mathcal{D}}} / \mathcal{T}_{\bm{A}}
      }
        ((\bm{E}, \overline{\bm{v}}^{\mathcal{D}})_{\bm{A}_{\mathcal{D}}}, \bm{W}_{ \overline{\bm{v}}^{\mathcal{D}} })
        (\bm{E}, \overline{\bm{t}}^{\mathcal{D}})_{\bm{A}_{\mathcal{D}}}
        \left(
          \mathcal{P}_{\mathrm{so}} \times \mathcal{T}_{\bm{A}}
        \right)
\end{align}

\subsection{Spin space group search}

Some rotations in $\mathcal{P}(\mathcal{S})$ may not be compatible with $\mathcal{T}_{\bm{A}_{\mathcal{D}}}$.
We write the subgroup of $\mathcal{S}$ which is compatible with $\mathcal{T}_{\bm{A}_{\mathcal{D}}}$ as $\mathcal{S}_{\mathcal{D}}$ and its point group as $\mathcal{P}_{\mathcal{D}}$.

\begin{align}
  \mathcal{P}_{\mathcal{D}}
    &\coloneqq \set{
        \bm{A}_{\mathcal{S}} \overline{\bm{R}}^{\mathcal{S}} \bm{A}_{\mathcal{S}}^{-1} \in \mathcal{P}(\mathcal{S})
      }{
        \bm{V}^{-1} \overline{\bm{R}}^{\mathcal{S}} \bm{V} \in \mathbb{Z}^{3 \times 3}
      }
\end{align}

\begin{align}
  \mathcal{S}_{\mathcal{D}}
    &\coloneqq \bigsqcup_{
          \overline{\bm{R}}^{\mathcal{S}} \in \bm{A}_{\mathcal{S}}^{-1} \mathcal{P}_{\mathcal{D}} \bm{A}_{\mathcal{S}}
      } \bigsqcup_{
          \left( \bm{E}, \overline{ \bm{t} }^{\mathcal{S}} \right)_{ \bm{A}_{\mathcal{S}} } \mathcal{T}_{\bm{A}_{\mathcal{D}}}
          \in \mathcal{T}_{\bm{A}_{\mathcal{S}}} / \mathcal{T}_{\bm{A}_{\mathcal{D}}}
      }
      \left(
        \overline{\bm{R}}^{\mathcal{S}},
        \overline{ \bm{v} }^{\mathcal{S}}(\overline{\bm{R}}^{\mathcal{S}})
      \right)_{ \bm{A}_{\mathcal{S}} }
      \left( \bm{E}, \overline{ \bm{t} }^{\mathcal{S}} \right)_{ \bm{A}_{\mathcal{S}} }
      \mathcal{T}_{\bm{A}_{\mathcal{D}}},
\end{align}

Finally, the coset decomposition of $\mathcal{G}$ by $\mathcal{G}_{\mathrm{st}}$ is finite.
The rotation parts of the coset representatives should belong to the point group of $\mathcal{F}(\mathcal{G})$.
The corresponding spin-rotation parts are determined by solving the Procrustes problem as well.
\begin{align}
  \mathcal{G}
    = \bigsqcup_{ \bm{R} \in \mathcal{P}(\mathcal{F}(\mathcal{G})) } ((\bm{R}, \bm{v}_{\bm{R}}), \bm{W}_{\bm{R}}) \mathcal{G}_{\mathrm{st}}
\end{align}
